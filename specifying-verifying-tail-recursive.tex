% vim: set tw=89 spell :
\documentclass[runningheads]{llncs}

\usepackage[T1]{fontenc}
\usepackage{listings}
\usepackage{graphicx}
\usepackage{lmodern}
\usepackage{colonequals}
\usepackage{amsmath}

\lstset{
  basicstyle=\ttfamily,
  extendedchars=true,
  literate=
    {∷}{{ $\coloncolon$ }}1 {→}{{ $\rightarrow$ }}1
    {ℕ}{{ $\mathbb{N}$ }}1 {≡}{{ $\equiv$ }}1
    {∀}{{ $\forall$ }}1 {λ}{{$\lambda$}}1
    {ˡ}{{ $\textsuperscript{l}$ }}1
    {ʳ}{{ $\textsuperscript{r}$ }}1
    ,
  inputencoding=utf8,
  captionpos=b,
}

\begin{document}

\title{Specifying and Verifying a Transformation of Recursive Functions into
       Tail-Recursive Functions}

\author{ Axel Suárez Polo\orcidID{0000-0002-8233-3751} \and \\
         José de Jesús Lavalle Martínez\orcidID{0000-0001-8652-3889} \and \\
         Iván Molina Rebolledo\orcidID{0000-0002-2224-7026}
    }

\authorrunning{A. Suárez et al.}

\institute{ Benemérita Universidad Autónoma de Puebla,\\
  Puebla, Puebla 72000, México}

\maketitle

\begin{abstract}

It is well known that some recursive functions admit a tail recursive counterpart which
have a more efficient time-complexity behavior. This paper presents a formal
specification and verification of such process. A monoid is used to generate a recursive
function and its tail-recursive counterpart. Also, the monoid properties are used to
prove extensional equality of both functions. In order to achieve this goal, the Agda
programming language and proof assistant is used to generate a parametrized module with a
monoid, via dependent types. This technique is exemplified with the length, reverse, and
indices functions over lists.

\keywords{    Dependent Types
         \and Formal Specification and Verification
         \and Tail Recursion
         \and Accumulation
         \and Program Transformation
         }
\end{abstract}

\section{Introduction}

Dependently typed programming languages provide an expressive system that allows both
programming and theorem proving. Agda is an implementation of such a language
\cite{bove2009brief}. Using these programming languages, it can be proven that two
functions return the same output when they receive the same input, which is a property
known as \emph{extensional equality} \cite{botta2021extensional}.

Programs can be developed using a transformational approach, where an initial program
whose correctness is easy to verify is written, and after that, it is transformed into a
more efficient program that preserves the same properties and semantics
\cite{pettorossi1993rules}. Proving that the transformed program works the same way as
the original program is usually done by using \emph{algebraic reasoning}
\cite{bird1996algebra}, but this can also be done using dependently typed programming
\cite{mu2008algebra}, with the advantage of the proof being verified by the compiler.

The \emph{accumulation} strategy is a well-known program transformation technique to
improve the efficiency of recursive functions \cite{bird1984promotion}. This technique is
the focus of this paper, in which dependently typed programming is used to develop an
strategy to prove extensional equality between the original recursive programs and their
tail-recursive counterparts.

\section{A simple example: list length}

Let's start with a simple example: a function to compute the length of a list. This
function can be defined recursively as follows:

\lstinputlisting[firstline=8, lastline=10]{code/Length.agda}

Nonetheless, this function requires space proportional to the length of the list. For
this reason, this program can be transformed into a tail-recursive function, which can be
optimized automatically by the compiler to use constant space
\cite{bauer2003compilation}. The transformed function is shown below:

\lstinputlisting[firstline=12, lastline=14]{code/Length.agda}

In this example, it is clear to see that for any list, both functions return the same
result, but this is a fact that is not encoded in Agda itself. However, we can encode
this fact in Agda using dependent function types and intensional equality:

\lstinputlisting[firstline=34, lastline=35]{code/Length.agda}

As we can see, the use of dependent types allows us to make reference to the first
argument in the return type of the function, and therefore, allowing us to state the fact
that both functions return the same result when called with the same list. The notion of
``sameness'' used here is the one of \emph{intensional equality}, which is an inductively
defined family of types \cite{dybjer1994inductive,mu2008algebra} defined as follows:

\begin{lstlisting}
data _≡_ {a} {A : Set a} (x : A) : A → Set a where
  instance refl : x ≡ x
\end{lstlisting}

What this means is that two terms are equal if they are exactly the same term.
Additionally, in Agda, if both terms reduce to the same term, we can state that they are
intensionally equal \cite{botta2021extensional}. For example, \lstinline{refl : 2 + 3 ≡
5}.

Returning to the example of the list length, we need to implement a function with that
signature, which we can do by pattern matching on the input list:

\begin{lstlisting}
length≡length-tail : ∀ (xs : List A)
                   → length xs ≡ length-tail xs 0
length≡length-tail [] = ?
length≡length-tail (x ∷ xs) = ?
\end{lstlisting}

We can proceed our proof by filling each of the holes\footnote{The \lstinline{?} symbols
are holes in our proof, which must be filled later to complete the proof, but are useful
to write the proof incrementally.}. The first case is trivial, because both function
calls reduce to to the same term:

\begin{align*}
  \lstinline{length []} &= \lstinline{0} & \text{(by definition)} \\
  \lstinline{length-tail [] 0} &= \lstinline{0} \\
\end{align*}

Therefore, we can fill the first hole with \lstinline{refl}:

\begin{center}
\begin{tabular}{c}
\begin{lstlisting}
length≡length-tail [] = refl
\end{lstlisting}
\end{tabular}
\end{center}

For the inductive case, we can reduce both terms instantiated with the argument, and
check what is necessary to prove. Note that this can be done automatically by querying
Agda and it's particularly useful when using the Agda mode in Emacs
\cite{wadler2018programming}. The reductions are shown below:

\begin{align*}
  \lstinline{length (x∷xs)} &= \lstinline{suc (length xs)} \\
  \lstinline{length-tail (x∷xs) 0} &= \lstinline{length-tail xs (suc 0)} \\
                                   &= \lstinline{length-tail xs 1} \\
\end{align*}

We need to prove that \lstinline{suc (length xs)≡length-tail xs 1}. This time, we cannot
simply use \lstinline{refl}, because both sides do not reduce to the same term. For this
reason, we can proceed to call this function recursively with the tail of the list. This
is justified because of the Curry-Howard correspondence, and the fact that we are making
a proof by induction. The result of the recursive call gives us the induction hypothesis:

\begin{lstlisting}
length≡length-tail (x ∷ xs) =
  let ind-h = length≡length-tail xs
   in ?
\end{lstlisting}

The type of \lstinline{ind-h} is \lstinline{length xs≡length-tail xs 0}. The left sides
of the induction hypothesis and what we are proving are almost the same. To make them
match, we can apply the \emph{congruence} property of equality, which has the following
type:

\begin{lstlisting}
cong : ∀ (f : A → B) {x y} → x ≡ y → f x ≡ f y
\end{lstlisting}

Applying this function to the induction hypothesis, we get the function below:

\begin{lstlisting}
length≡length-tail (x ∷ xs) =
  let ind-h = length≡length-tail xs
      suc-cong = cong suc ind-h
   in ?
\end{lstlisting}

The new term \lstinline{suc-cong} has the type:

\begin{center}
\begin{tabular}{c}
\begin{lstlisting}
suc (length xs) ≡ suc (length-tail xs 0)
\end{lstlisting}
\end{tabular}
\end{center}

The proof can be completed by making use of \lstinline{transitivity}, which is
represented using dependent types as follows:

\begin{center}
\begin{tabular}{c}
\begin{lstlisting}
trans : ∀ {x y z} → x ≡ y → y ≡ z → x ≡ z
\end{lstlisting}
\end{tabular}
\end{center}

Therefore, now our proof is:

\begin{lstlisting}
length≡length-tail (x ∷ xs) =
  let ind-h = length≡length-tail xs
      suc-cong = cong suc ind-h
   in trans suc-cong ?
\end{lstlisting}

The type of the term required to fill the hole is:

\begin{center}
\begin{tabular}{c}
\begin{lstlisting}
suc (length-tail xs 0) ≡ length-tail xs 1
\end{lstlisting}
\end{tabular}
\end{center}

We need to ``pull'' the \lstinline{1} from the accumulator somehow, and convert it to a
\lstinline{suc} call. We can extract this new goal into a helper function:

\lstinputlisting[firstline=30, lastline=31]{code/Length.agda}

We can try to prove this goal by straightforward induction over the list, but we reach a
dead end:

\begin{lstlisting}
length-pull [] = refl
length-pull (x ∷ xs) = ?
\end{lstlisting}

The base case is trivial, following the definitions of the function, both terms reduce to
\lstinline{1}. The problem is the inductive case, which reduces as follows:

\begin{align*}
  \lstinline{suc (length-tail (x∷xs) 0)} &= \lstinline{suc (length-tail xs (suc 0))} \\
                                         &= \lstinline{suc (length-tail xs 1)} \\
  \lstinline{length-tail (x∷xs) 1} &= \lstinline{length-tail xs (suc 1)} \\
                                   &= \lstinline{length-tail xs 2} \\
\end{align*}

So, we are left with the following goal, which is very similar to the one we started
with:

\begin{center}
\begin{tabular}{c}
\begin{lstlisting}
suc (length-tail xs 1) ≡ length-tail xs 2
\end{lstlisting}
\end{tabular}
\end{center}

We could try to prove this proposition by straightforward induction too, but that would
require us to prove the proposition for the next values \lstinline{2} and \lstinline{3},
and so on, \emph{ad infinitum}.

To solve this issue, we can use a \emph{generalization} strategy to prove this inductive
property \cite{abdali1984generalization}. The generalized property will allows us to
vary the value of the accumulator in the different cases of the inductive proof, but we
will need to introduce another variable for it. It is important to note that after
processing the first \lstinline{n} items of the list, we will get
\lstinline{n + length-tail xs 0} on the left side and \lstinline{length-tail xs n} on the
right one. Combining the generalization strategy and this fact, we can see that the
property we have to prove is:

\lstinputlisting[firstline=22, lastline=24]{code/Length.agda}

This function can be proved by induction over the list:

\begin{lstlisting}
length-pull-generalized [] n p = refl
length-pull-generalized (x ∷ xs) n p = ?
\end{lstlisting}

The base case is trivial, because replacing the \lstinline{xs} argument with
\lstinline{[]}, and following a single reduction step on both sides, a common term
\lstinline{n + p} is reached.

The inductive case is more interesting, reducing both sides of the equation proceeds as
follows:

\begin{align*}
  \lstinline{n + length-tail (x∷xs) p} &= \lstinline{n + length-tail xs (suc p)} \\
  \lstinline{length-tail (x∷xs) (n + p)} &= \lstinline{length-tail xs (suc (n + p))}
\end{align*}

We can see that we have pretty much the induction hypothesis, with the only difference
being the accumulating parameter \lstinline{p}. Fortunately, as we have generalized the
proposition, we can pick a value for \lstinline{p} when using the induction
hypothesis:

\begin{lstlisting}
length-pull-generalized (x ∷ xs) n p =
  length-pull-generalized xs n (suc p)
\end{lstlisting}

This takes us closer to the goal we want to prove. Unfortunately, we are left with the
following type after performing the substitution of \lstinline{p} with \lstinline{suc p}:

\begin{lstlisting}
n + length-tail xs (suc p) ≡ length-tail xs (n + suc p)
\end{lstlisting}

This is almost what we want, with the exception of \lstinline{suc (n + p)} not being
equal to \lstinline{n + suc p}. However, these two terms are indeed equal, but not
definitionally, because the plus function is defined by induction on the first argument,
and not on the second one:

\begin{lstlisting}
_+_ : Nat → Nat → Nat
zero  + m = m
suc n + m = suc (n + m)
\end{lstlisting}

Therefore, applying reduction steps does not allow Agda to deduce the equality of these
two terms. Fortunately, the fact that these terms are equal can be easily proved
inductively as follows:

\begin{lstlisting}
+-suc : ∀ m n → m + suc n ≡ suc (m + n)
+-suc zero    n = refl
+-suc (suc m) n = cong suc (+-suc m n)
\end{lstlisting}

The remaining step is to ``replace'' the \lstinline{suc (n + p)} term with
\lstinline{n + suc p}. This could be done by using a function generalizing over the
accumulator, but Agda provides the \lstinline{rewrite} construct, that gives us that
functionality with minimal effort:

\lstinputlisting[firstline=26, lastline=28]{code/Length.agda}

We make use of the \emph{symmetric} property of equality in the rewriting step, which
allows us to flip the sides of the equality:

\begin{center}
\begin{tabular}{c}
\begin{lstlisting}
sym : ∀ {x y} → x ≡ y → y ≡ x
\end{lstlisting}
\end{tabular}
\end{center}

With all this in place, we can finally prove the remaining goals, giving as a result the
complete proof:

\lstinputlisting[firstline=22, lastline=41]{code/Length.agda}

\section{Another example: list reverse}

The list reversal function follows a similar pattern to the one we have seen before:

\lstinputlisting[firstline=7, lastline=13]{code/Rev.agda}

It should not come as a surprise that the equality proof is very similar too:

\lstinputlisting[firstline=17, lastline=39]{code/Rev.agda}

There are minor variations in the function signatures and the order of the parameters,
but the structure is identical:

\begin{itemize}
  \item Start proving by induction on the list.
  \item Fill the base case with \lstinline{refl}.
  \item Take the induction hypothesis by using a recursive call.
  \item Apply \emph{an operator} to both sides of the equality, using \lstinline{cong}.
  \item Create a function to pull the accumulator, and prove it using a generalized
        version of this function that allows varying the accumulator.
  \item Compose the two equalities using the \lstinline{trans} function.
\end{itemize}

\section{Generalization}

Starting from the function definitions, we can see that they follow the same recursive
pattern, we can write this pattern in Agda, which is just an specialization of a
\lstinline{fold} function \cite{hutton1999tutorial,meijer1991functional}:

\lstinputlisting[firstline=21, lastline=23]{code/GenericBasic.agda}

\begin{itemize}
  \item \lstinline{R} is the result type of the function.
  \item \lstinline{empty} is the term to return when the list is empty.
  \item \lstinline{f} is a function to transform each element of the list into the result
        type.
  \item \lstinline{<>} is the function to combine the current item and the recursive
        result.
\end{itemize}

In the case of the \lstinline{length} function, the result type is \mathbb{N}, the
natural numbers; \lstinline{empty} is \lstinline{0}; the function to transform each
element is a constant function that ignores its argument and returns \lstinline{1}; and
the function to combine the current item and the result of the recursive call is the
addition function.

For the \lstinline{reverse} function, the result type is the same type as the original
list, \lstinline{List A}; \lstinline{empty} is the empty list; the function to
transform each element creates just a singleton list from its parameter; and the function
to combine the current transformed item and the result of the recursive call, is the
flipped concatenation function. The flipping is necessary to make the function
concatenate its first argument to the right:

\begin{align*}
  \lstinline{reduce (x∷xs)} &= \lstinline{(λa→a∷[]) x <> reduce xs} \\
                            &= \lstinline{(x∷[]) <> reduce xs} \\
                            &= \lstinline{(λxs ys→ys ++ xs) (x∷[]) (reduce xs)} \\
                            &= \lstinline{reduce xs ++ (x∷[])} \\
\end{align*}

The functions that follow this pattern, can be defined in a tail-recursive way as
follows:

\lstinputlisting[firstline=25, lastline=27]{code/GenericBasic.agda}

We can check manually that this function matches the tail-recursive definition in the
case of the \lstinline{reverse} function:

\begin{align*}
  \lstinline{reduce-tail (x∷xs) r}
    &= \lstinline{reduce-tail xs (r <> (λa→a∷[]) x)} \\
    &= \lstinline{reduce-tail xs (r <> (x∷[])} \\
    &= \lstinline{... xs ((λxs ys→ys ++ xs) r (x∷[]))} \\
    &= \lstinline{reduce-tail xs ((x∷[]) ++ r)} \\
    &= \lstinline{reduce-tail xs (x∷r)} \\
\end{align*}

Now we can proceed to prove that these two functions are extensionally equal in the
general case. The proof follows the same pattern as the one for the \lstinline{length}
function:

\lstinputlisting[firstline=49, lastline=56]{code/GenericBasic.agda}

We make use of a piece of syntactic sugar called \emph{sections}, which allows us to
write the function \lstinline{(λr→f x <> r)} as \lstinline{(f x <>_)}. Apart from that,
the proof is identical to the ones we have seen before.

However, to prove the accumulator pulling function, we need to use a different strategy.
We are required to prove that:

\lstinputlisting[firstline=37, lastline=40]{code/GenericBasic.agda}

To do this, we can prove this proposition by induction over the list, which requires us
to prove the proposition when \lstinline{xs} is \lstinline{[]}:

\begin{align*}
  \lstinline{r <> reduce-tail [] empty} &= \lstinline{r <> empty} \\
  \lstinline{reduce-tail [] (empty <> r)} &= \lstinline{empty <> r} \\
\end{align*}

So we are required to prove that \lstinline{r <> empty≡empty <> r}. We could require the
\lstinline{<>} function to be commutative, but we can ``ask for less'' by just requiring
\lstinline{empty} to be a left and right identity for \lstinline{<>}, this is expressed
in Agda as:

\begin{center}
\begin{tabular}{c}
\begin{lstlisting}
<>-identityˡ : ∀ (r : R) → empty <> r ≡ r
<>-identityʳ : ∀ (r : R) → r <> empty ≡ r
\end{lstlisting}
\end{tabular}
\end{center}

This way, we can use those identities to rewrite our goals, and make them match over the
term \lstinline{r}, and then, complete the base case using the trivial equality proof
\lstinline{refl}:

\lstinputlisting[firstline=41, lastline=43]{code/GenericBasic.agda}

The inductive case goal is:

\begin{align*}
  \lstinline{r <}&\lstinline{> reduce-tail (x∷xs) empty} \\
    &= \lstinline{r <> reduce-tail xs (empty <> f x)} \\
  \lstinline{red}&\lstinline{uce-tail (x∷xs) (empty <> r)} \\
    &= \lstinline{reduce-tail xs ((empty <> r) <> f x)} \\
\end{align*}

Which cannot be proven directly by straightforward induction, as we have seen before, but
at least we can simplify it by using the left identity property over
\lstinline{empty <> f x} and then over \lstinline{empty <> r}:

\lstinputlisting[firstline=44, lastline=47]{code/GenericBasic.agda}

Finally, we just need to prove the generalized accumulation pulling function, which has
the following type signature:

\lstinputlisting[firstline=29, lastline=31]{code/GenericBasic.agda}

Note that the base case is trivial, and it is quite similar to the ones we have already
proved, so we are going to focus in the inductive case. Following the same kind of
reductions we have been doing before, we can see that our goal is:

\begin{align*}
  \lstinline{r <}&\lstinline{> reduce-tail (x∷xs) s} \\
    &= \lstinline{r <> reduce-tail xs (s <> f x)} \\
  \lstinline{red}&\lstinline{uce-tail (x∷xs) (s <> r)} \\
    &= \lstinline{reduce-tail xs ((r <> s) <> f x)} \\
\end{align*}

Following the generalization strategy, we have to call the function recursively,
replacing the \lstinline{s} by \lstinline{s <> f x}, which almost gives us what we want,
except that the right hand side accumulator is associated wrongly.

\begin{center}
\begin{tabular}{c}
\begin{lstlisting}
r <> reduce-tail xs (s <> f x)
   ≡ reduce-tail xs (r <> (s <> f x))
\end{lstlisting}
\end{tabular}
\end{center}

Associativity is indeed the last property we need the \lstinline{<>} function to satisfy.
This can be expressed in Agda straightforwardly as:

\begin{center}
\begin{tabular}{c}
\begin{lstlisting}
<>-assoc : ∀ (r s t : R)
         → (r <> s) <> t ≡ r <> (s <> t)
\end{lstlisting}
\end{tabular}
\end{center}

Which helps us complete the proof:

\lstinputlisting[firstline=32, lastline=35]{code/GenericBasic.agda}

All of these properties match the definition of a monoid. We can complete the
formalization and encapsulate it in a ready to use parametrized module, using the
standard library definition of a monoid:

\lstinputlisting[firstline=4, lastline=19]{code/GenericBasic.agda}

\section{Examples}


\section{Conclusions}

\subsubsection{Acknowledgements} Please place your acknowledgments at
the end of the paper, preceded by an unnumbered run-in heading (i.e.
3rd-level heading).


\bibliographystyle{splncs04}
\bibliography{refs}
\end{thebibliography}
\end{document}
