% vim: set tw=89 spell :
\documentclass{cys}

%\usepackage[T1]{fontenc}
\usepackage{fontspec}

\usepackage{graphicx}
%\usepackage{lmodern}
\usepackage{colonequals}
\usepackage{amsmath}
\usepackage[english]{babel}
\usepackage{hyperref}
\usepackage{color}
\usepackage[scaled=0.85]{beramono}
\renewcommand\UrlFont{\color{blue}\rmfamily}
\usepackage{unicode-math}
\usepackage{verbatim}
\usepackage{fancyvrb}
\usepackage{fvextra}
\usepackage{microtype}

% Temporal font
\setmonofont{Fira Code}

\fvset{
  breaklines=true,
  breakindent=2em,
}

% Inline code for math environments
\newcommand{\MVerb}[1]{\hbox{\Verb{#1}}}

\begin{document}

\title{Specifying and Verifying a Transformation of Recursive Functions into
       Tail-Recursive Functions}

\author{Axel Suárez Polo$^1$, José de Jesús Lavalle Martínez$^1$, Iván Molina Rebolledo$^1$}

\affil{
$^1$ Benemérita Universidad Autónoma de Puebla, Puebla, \authorcr
Mexico
\authorcr \authorcr
author1@xxx.xx, author2@xxx.xx
\authorcr  \authorcr
}


%\titlerunning{Transforming recursive functions into tail-recursive functions}
%\authorrunning{A. Suárez et al.}

%\institute{ Benemérita Universidad Autónoma de Puebla,\\
%  Puebla, Puebla 72000, México}

\maketitle

\begin{abstract}
It is well known that some recursive functions admit a tail recursive counterpart which
have a more efficient time-complexity behavior. This paper presents a formal
specification and verification of such process. A monoid is used to generate a recursive
function and its tail-recursive counterpart. Also, the monoid properties are used to
prove extensional equality of both functions. In order to achieve this goal, the Agda
programming language and proof assistant is used to generate a parametrized module with a
monoid, via dependent types. This technique is exemplified with the length, reverse, and
indices functions over lists.
\end{abstract}

\begin{keywords}
Dependent Types, Formal Specification and Verification, Tail Recursion, Accumulation,
Program Transformation
\end{keywords}

\section{Introduction}
\label{sec:introduction}

Dependently typed programming languages provide an expressive system that allows both
programming and theorem proving. Agda is an implementation of such a kind of language
\cite{bove2009brief}. Using these programming languages, it can be proved that two
functions return the same output when they receive the same input, which is a property
known as \emph{extensional equality} \cite{botta2021extensional}.

Programs can be developed using a transformational approach, where an initial program
whose correctness is easy to verify is written, and after that, it is transformed into a
more efficient program that preserves the same properties and semantics
\cite{pettorossi1993rules}. Proving that the transformed program works the same way as
the original program is usually done by using \emph{algebraic reasoning}
\cite{bird1996algebra}, but this can also be done using dependently typed programming
\cite{mu2008algebra}, with the advantage of the proof being verified by the compiler.

The \emph{accumulation} strategy is a well-known program transformation technique to
improve the efficiency of recursive functions \cite{bird1984promotion}. This technique is
the focus of this paper, in which dependently typed programming is used to develop a
strategy to prove extensional equality between the original recursive programs and their
tail-recursive counterparts.

The source code of this paper is available at
\url{https://github.com/ggzor/specifying-verifying-tail-recursion}.

\section{A simple example: list length}

Let us start with a simple example: a function to compute the length of a list. This
function can be defined recursively as follows:

\VerbatimInput[firstline=8, lastline=10]{code/Length.agda}

Nonetheless, this function requires space proportional to the length of the list due to
the recursive calls. This program can be transformed into a tail-recursive function,
which can be optimized automatically by the compiler to use constant space
\cite{bauer2003compilation}. The transformed function is shown below:

\VerbatimInput[firstline=12, lastline=14]{code/Length.agda}

In this example, it is clear to see that both functions return the same result for every
possible list we provide as input. This fact can be represented in Agda using dependent
function types:

\VerbatimInput[firstline=34, lastline=35]{code/Length.agda}

The notion of ``sameness'' used here is the one of \emph{intensional equality}, which is
an inductively defined family of types \cite{dybjer1994inductive,mu2008algebra} with the
following definition:

\begin{Verbatim}
data _≡_ {a} {A : Set a} (x : A) : A → Set a where
  instance refl : x ≡ x
\end{Verbatim}

This means that two terms are equal if they are exactly the same term. Additionally, in
Agda, if both terms reduce to the same term, we can state that they are intensionally
equal. For example, \Verb{refl : 2 + 3 ≡ 5}.

This notion of equality together with the addition of the universal quantifier, allows us
to state a kind of equality for functions, known as \emph{point-wise equality} or
\emph{extensional equality} \cite{botta2021extensional}.

To prove extensional equality for the length functions, we can proceed inductively over
the list, which has the \Verb{[]} and \Verb{x∷xs} cases\footnote{The
\Verb{?} symbols are holes, which must be filled later to complete the proof, but
are useful to write the proof incrementally.}:

\begin{Verbatim}
length≡length-tail : ∀ (xs : List A)
                   → length xs ≡ length-tail xs 0
length≡length-tail [] = ?
length≡length-tail (x ∷ xs) = ?
\end{Verbatim}

The base case is trivial, because both sides of the equality in the result type reduce to
the same term:
\begin{align*}
  \MVerb{length []} &= \MVerb{0} & \text{(by definition)} \\
  \MVerb{length-tail [] 0} &= \MVerb{0}
\end{align*}

Therefore, we can fill the first hole in our proof with \Verb{refl}:
\begin{center}
\begin{Verbatim}
length≡length-tail [] = refl
\end{Verbatim}
\end{center}

For the inductive case, we can reduce both sides of the equality instantiated with the
argument, and check what is necessary to prove. Note that this can be done automatically
by querying Agda, and it is particularly useful when using the Agda mode in Emacs
\cite{wadler2018programming}. The reductions are shown below and follow from the
definition:
\begin{align*}
  \MVerb{length (x∷xs)} &= \MVerb{suc (length xs)} \\
  \MVerb{length-tail (x∷xs) 0} &= \MVerb{length-tail xs (suc 0)} \\
                                   &= \MVerb{length-tail xs 1} \\
\end{align*}

We need to prove that \Verb{suc (length xs)≡length-tail xs 1}. This time, we cannot
simply use \Verb{refl}, because both sides do not reduce to the same term. For this
reason, we can proceed to call this function recursively with the tail of the list. This
is justified because of the Curry-Howard correspondence, and the fact that we are making
a proof by induction. The result of the recursive call gives us the induction hypothesis:

\begin{Verbatim}
length≡length-tail (x ∷ xs) =
  let ind-h = length≡length-tail xs
   in ?
\end{Verbatim}

The type of \Verb{ind-h} is \Verb{length xs≡length-tail xs 0}. The left sides
of the induction hypothesis and what we are proving are almost the same. To make them
match, we can apply the \emph{congruence} property of equality, which has the following
type:

\begin{center}
\begin{Verbatim}
cong : ∀ (f : A → B) {x y} → x ≡ y → f x ≡ f y
\end{Verbatim}
\end{center}

Applying this function to the induction hypothesis, we get the function below:

\begin{Verbatim}
length≡length-tail (x ∷ xs) =
  let ind-h = length≡length-tail xs
      suc-cong = cong suc ind-h
   in ?
\end{Verbatim}

The \Verb{suc-cong} term has the type:

\begin{center}
\begin{Verbatim}
suc (length xs) ≡ suc (length-tail xs 0)
\end{Verbatim}
\end{center}

As we can see the left sides match, so we can change our goal to prove that the right
side of \Verb{suc-cong} is equal to the right side of the goal; by making use of the
transitive property of equality, which has the following type in Agda:

\begin{center}
\begin{Verbatim}
trans : ∀ {x y z} → x ≡ y → y ≡ z → x ≡ z
\end{Verbatim}
\end{center}

Therefore, now our proof is:

\begin{Verbatim}
length≡length-tail (x ∷ xs) =
  let ind-h = length≡length-tail xs
      suc-cong = cong suc ind-h
   in trans suc-cong ?
\end{Verbatim}

The type of the term required to fill the hole is:

\begin{center}
\begin{Verbatim}
suc (length-tail xs 0) ≡ length-tail xs 1
\end{Verbatim}
\end{center}

We need to ``pull'' the \Verb{1} from the accumulator somehow, and convert it to a
\Verb{suc} call. We can extract this new goal into a helper function:

\VerbatimInput[firstline=30, lastline=31]{code/Length.agda}

We can try to prove this goal by straightforward induction over the list, but we reach a
dead end:

\begin{Verbatim}
length-pull [] = refl
length-pull (x ∷ xs) = ?
\end{Verbatim}

The base case is trivial, following the definitions of the function, both terms reduce to
\Verb{1}. The problem is the inductive case, which reduces as follows:

\begin{align*}
  \MVerb{suc (length-tail (x∷xs) 0)} &= \MVerb{suc (length-tail xs (suc 0))} \\
                                         &= \MVerb{suc (length-tail xs 1)} \\
  \MVerb{length-tail (x∷xs) 1} &= \MVerb{length-tail xs (suc 1)} \\
                                   &= \MVerb{length-tail xs 2} \\
\end{align*}

So, we are left with the following goal, which is very similar to the one we started
with:

\begin{center}
\begin{Verbatim}
suc (length-tail xs 1) ≡ length-tail xs 2
\end{Verbatim}
\end{center}

We could try to prove this proposition by straightforward induction too, but that would
require to prove a similar proposition for the next values \Verb{2} and
\Verb{3}, and so on.

To solve this issue, we can use a \emph{generalization} strategy to prove this inductive
property \cite{abdali1984generalization}. The generalized property will allow us to
vary the value of the accumulator in the different cases of the inductive proof, but we
will need to introduce another variable for it. It is important to note that after
processing the first \Verb{n} items of the list, we will get
\Verb{n + length-tail xs 0} on the left side and \Verb{length-tail xs n} on the
right one. Combining the generalization strategy and this fact, we can see that the
property we have to prove is:

\VerbatimInput[firstline=22, lastline=24]{code/Length.agda}

This function can be proved by induction over the list:

\begin{Verbatim}
length-pull-generalized [] n p = refl
length-pull-generalized (x ∷ xs) n p = ?
\end{Verbatim}

The base case is trivial, because replacing the \Verb{xs} argument with
\Verb{[]}, and following a single reduction step on both sides, the common term
\Verb{n + p} is reached.

The inductive case is more interesting. Reducing both sides of the equation proceeds as
follows:
\begin{align*}
  \MVerb{n + length-tail (x∷xs) p} &= \MVerb{n + length-tail xs (suc p)} \\
  \MVerb{length-tail (x∷xs) (n + p)} &= \MVerb{length-tail xs (suc (n + p))}
\end{align*}

We can see that we have pretty much the induction hypothesis, with the only difference
being the accumulating parameter \Verb{p}. Nevertheless, as we have generalized the
proposition, we can pick a value for \Verb{p} when using the induction hypothesis:

\begin{Verbatim}
length-pull-generalized (x ∷ xs) n p =
  length-pull-generalized xs n (suc p)
\end{Verbatim}

This takes us closer to the goal we want to prove. Unfortunately, we are left with the
following goal after performing the substitution of \Verb{p} with \Verb{suc p}:

\begin{Verbatim}
n + length-tail xs (suc p) ≡ length-tail xs (n + suc p)
\end{Verbatim}

This is almost what we want, except for \Verb{suc (n + p)} not being
equal to \Verb{n + suc p}. However, these two terms are indeed equal, but not
definitionally, because the plus function is defined by induction on the first argument,
and not on the second one:

\begin{Verbatim}
_+_ : Nat → Nat → Nat
zero  + m = m
suc n + m = suc (n + m)
\end{Verbatim}

Therefore, applying reduction steps does not allow Agda to deduce the equality of these
two terms. Fortunately, the fact that these terms are equal can be easily proved
inductively as follows:

\begin{Verbatim}
+-suc : ∀ m n → m + suc n ≡ suc (m + n)
+-suc zero    n = refl
+-suc (suc m) n = cong suc (+-suc m n)
\end{Verbatim}

The remaining step is to ``replace'' the \Verb{suc (n + p)} term with
\Verb{n + suc p}. Agda provides the \Verb{rewrite} construct to perform this
transformation:

\VerbatimInput[firstline=26, lastline=28]{code/Length.agda}

We make use of the \emph{symmetric} property of equality in the rewriting step, which
allows us to flip the sides of the equality:

\begin{center}
\begin{Verbatim}
sym : ∀ {x y} → x ≡ y → y ≡ x
\end{Verbatim}
\end{center}

With all this in place, we can finally prove the remaining goals, giving as a result the
complete proof:

\VerbatimInput[firstline=22, lastline=41]{code/Length.agda}

\section{Another example: list reverse}

The list reversal function follows a similar pattern to the one we have seen before:

\VerbatimInput[firstline=7, lastline=13]{code/Rev.agda}

It should not come as a surprise that the equality proof is very similar too:

\VerbatimInput[firstline=17, lastline=39]{code/Rev.agda}

There are minor variations in the function signatures and the order of the parameters,
but the structure is identical:

\begin{itemize}
  \item Start proving by induction on the list.
  \item Fill the base case with \Verb{refl}.
  \item Take the inductive hypothesis by using a recursive call.
  \item Apply \emph{an operator} to both sides of the equality, using \Verb{cong}.
  \item Create a function to pull the accumulator, and prove it using a generalized
        version of this function that allows varying the accumulator.
  \item Compose the two equalities using the \Verb{trans} function.
\end{itemize}

\section{Generalization}

Starting from the function definitions, we can see that they follow the same recursive
pattern, we can write this pattern in Agda, which is just a specialization of a
\Verb{fold} function \cite{hutton1999tutorial,meijer1991functional}:

\VerbatimInput[firstline=21, lastline=23]{code/GenericBasic.agda}

\noindent
where

\begin{itemize}
  \item \Verb{R} is the result type of the function.
  \item \Verb{empty} is the term to return when the list is empty.
  \item \Verb{f} is a function to transform each element of the list into the result
        type.
  \item \Verb{<>} is the function to combine the current item and the recursive
        result.
\end{itemize}

In the case of the \Verb{length} function, the result type is $\mathbb{N}$, the
natural numbers; \Verb{empty} is \Verb{0}; the function to transform each
element is a constant function that ignores its argument and returns \Verb{1}; and
the function to combine the current item and the result of the recursive call is the
addition function.

For the \Verb{reverse} function, the result type is the same type as the original
list, \Verb{List A}; \Verb{empty} is the empty list; the function to
transform each element creates just a singleton list from its parameter; and the function
to combine the current transformed item and the result of the recursive call, is the
flipped concatenation function. The flipping is necessary to make the function
concatenate its first argument to the right:

\begin{align*}
  \MVerb{reduce (x∷xs)} &= \MVerb{(λa→a∷[]) x <> reduce xs} \\
                            &= \MVerb{(x∷[]) <> reduce xs} \\
                            &= \MVerb{(λxs ys→ys ++ xs) (x∷[]) (reduce xs)} \\
                            &= \MVerb{reduce xs ++ (x∷[])} \\
\end{align*}

The functions that follow this pattern, can be defined in a tail-recursive way as
follows:

\VerbatimInput[firstline=25, lastline=27]{code/GenericBasic.agda}

We can check manually that this function matches the tail-recursive definition in the
case of the \Verb{reverse} function:

\begin{align*}
  \MVerb{reduce-tail (x∷xs) r}
    &= \MVerb{reduce-tail xs (r <> (λa→a∷[]) x)} \\
    &= \MVerb{reduce-tail xs (r <> (x∷[])} \\
    &= \MVerb{... xs ((λxs ys→ys ++ xs) r (x∷[]))} \\
    &= \MVerb{reduce-tail xs ((x∷[]) ++ r)} \\
    &= \MVerb{reduce-tail xs (x∷r)} \\
\end{align*}

Now we can proceed to prove that these two functions are extensionally equal in the
general case. The proof follows the same pattern as the one for the \Verb{length}
function:

\VerbatimInput[firstline=49, lastline=56]{code/GenericBasic.agda}

We make use of a piece of syntactic sugar called \emph{sections}, which allows us to
write the function \Verb{(λr→f x <> r)} as \Verb{(f x <>_)}. Apart from that,
the proof is identical to the ones we have seen before.

However, to prove the accumulator pulling function, we need to use a different strategy.
We are required to prove that:

\VerbatimInput[firstline=37, lastline=40]{code/GenericBasic.agda}

To do this, we can prove this proposition by induction over the list, which requires us
to prove the proposition when \Verb{xs} is \Verb{[]}:

\begin{align*}
  \MVerb{r <> reduce-tail [] empty} &= \MVerb{r <> empty} \\
  \MVerb{reduce-tail [] (empty <> r)} &= \MVerb{empty <> r} \\
\end{align*}

So we are required to prove that \Verb{r <> empty≡empty <> r}. We could require the
\Verb{<>} function to be commutative, but we can ``ask for less'' by just requiring
\Verb{empty} to be a left and right identity for \Verb{<>}, this is expressed
in Agda as:

\begin{center}
\begin{Verbatim}
<>-identityˡ : ∀ (r : R) → empty <> r ≡ r
<>-identityʳ : ∀ (r : R) → r <> empty ≡ r
\end{Verbatim}
\end{center}

This way, we can use those identities to rewrite our goals, and make them match over the
term \Verb{r}, and then, complete the base case using the trivial equality proof
\Verb{refl}:

\VerbatimInput[firstline=41, lastline=43]{code/GenericBasic.agda}

The inductive case goal is:

\begin{align*}
  \MVerb{r <}&\MVerb{> reduce-tail (x∷xs) empty} \\
    &= \MVerb{r <> reduce-tail xs (empty <> f x)} \\
  \MVerb{red}&\MVerb{uce-tail (x∷xs) (empty <> r)} \\
    &= \MVerb{reduce-tail xs ((empty <> r) <> f x)} \\
\end{align*}

Which cannot be proved directly by straightforward induction, as we have seen before, but
at least we can simplify it by using the left identity property over
\Verb{empty <> f x} and then over \Verb{empty <> r}:

\VerbatimInput[firstline=44, lastline=47]{code/GenericBasic.agda}

Finally, we just need to prove the generalized accumulation pulling function, which has
the following type signature:

\VerbatimInput[firstline=29, lastline=31]{code/GenericBasic.agda}

Note that the base case is trivial, and it is quite similar to the ones we have already
proved, so we are going to focus on the inductive case. Following the same kind of
reductions we have been doing before, we can see that our goal is:

\begin{align*}
  \MVerb{r <}&\MVerb{> reduce-tail (x∷xs) s} \\
    &= \MVerb{r <> reduce-tail xs (s <> f x)} \\
  \MVerb{red}&\MVerb{uce-tail (x∷xs) (s <> r)} \\
    &= \MVerb{reduce-tail xs ((r <> s) <> f x)} \\
\end{align*}

Following the generalization strategy, we have to call the function recursively,
replacing the \Verb{s} by \Verb{s <> f x}, which almost gives what it is
required, except that the right hand side accumulator is associated wrongly.

\begin{center}
\begin{Verbatim}
r <> reduce-tail xs (s <> f x)
   ≡ reduce-tail xs (r <> (s <> f x))
\end{Verbatim}
\end{center}

Associativity is indeed the last property that the \Verb{<>} function needs to
satisfy. This can be expressed in Agda straightforwardly as:

\begin{center}
\begin{Verbatim}
<>-assoc : ∀ (r s t : R)
         → (r <> s) <> t ≡ r <> (s <> t)
\end{Verbatim}
\end{center}

Which helps us complete the proof:

\VerbatimInput[firstline=32, lastline=35]{code/GenericBasic.agda}

All of these properties match the definition of a monoid. We can complete the
formalization and encapsulate it in a ready to use parametrized module, using the
standard library definition of a monoid:

\VerbatimInput[firstline=4, lastline=19]{code/GenericBasic.agda}

\section{Using the module with the examples}

With the module in place, we can start using it to derive the recursive function, the
tail-recursive counterpart, and the proof that both functions are extensionally equal.

The length function uses the usual sum monoid over the natural numbers:

\VerbatimInput[firstline=9, lastline=14]{code/Instances.agda}

The reverse function requires us to create an instance of a flipped monoid for
\Verb{++}, which can be done with the already defined properties for list
concatenation, but flipping them when necessary.

\VerbatimInput[firstline=32, lastline=41]{code/Instances.agda}

Finally, the \Verb{indices} function also requires us to create a custom monoid. The
original indices function specialized for lists of natural number is the following:

\VerbatimInput[firstline=7, lastline=11]{code/Indices.agda}

The monoid for this function has the following operation and identity element:

\VerbatimInput[firstline=67, lastline=75]{code/Instances.agda}

\section{Conclusions}

A technique to prove extensional equality between a recursive function and its
tail-recursive counterpart has been presented, along with an Agda module to generate the
functions and the proof automatically from an arbitrary monoid. The tail-recursive
function generally improves the time complexity of the original recursive function and
opens the possibility of performing tail-call optimization by the compiler, leading to a
more space efficient function execution
\cite{bauer2003compilation,rubio2017introduction}.

There are some caveats with this technique which are exemplified by the
\Verb{indices} function. Even though the generated function avoids mapping over the
entire recursive call result, it introduces inefficiency by doing nested concatenations
to the left, which leads to quadratic time complexity. This could be solved by using
higher order functions as the accumulating monoid \cite{hughes1986novel}, but proving the
corresponding monoid laws will require to be able to transform extensional equality to
intensional equality, which is not possible in Agda without using \emph{cubical type
theory} \cite{botta2021extensional,vezzosi2021cubical}, but that is out of the scope of
this paper.

Further work can be done in order to generalize this result to arbitrary \emph{recursive
data types} and \emph{recursion schemes} \cite{meijer1991functional}.

% OBLIGATORY: use BIBTEX formatting!
\small{
\bibliographystyle{cys}
\bibliography{refs}
}
\normalsize

\begin{biography}[]{} % Leave this section empty
\end{biography}

\end{document}
